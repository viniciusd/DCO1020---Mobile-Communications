\documentclass[journal,11pt,twocolumn]{IEEEtran}
\usepackage[portuguese]{babel}
\usepackage[utf8]{inputenc}
\usepackage[dvips]{graphicx}
\usepackage{amsmath,amsfonts,amssymb}
\usepackage{makecell}
\usepackage[normal]{subfigure}
\usepackage{array,colortbl}
\usepackage{amsmath}
\usepackage{mathrsfs}
\usepackage[sort]{cite}
\usepackage[none]{hyphenat}
\usepackage{minted}
\usepackage{url}
\usepackage[section]{placeins}
% Those packages were originally here, but I don't think we are gonna need them, so I commented them out
%\usepackage[center,small]{caption}
%\usepackage{color}
%\usepackage{colortbl}
%\usepackage{comment}
%\usepackage{float}
%\usepackage{flushend}
%\usepackage[mathfrak]{mathpi}
%\usepackage{morefloats}
%\usepackage{multicol}
%\usepackage{multirow}
%\usepackage{psfrag}
%\usepackage{rangecite}
%\usepackage{tabularx}
%\usepackage{threeparttable}

\setminted{breaklines=true}

\sloppy
\newcommand{\goodgap}{%
\hspace{\subfigtopskip}%
\hspace{\subfigbottomskip}}
\newcounter{mytempeqncnt}

\begin{document}

% paper title
\title{Calculadora LTE de downlink}


\author{Vinícius Dantas de Lima Melo
    \thanks{Este trabalho foi como parte do curso de Comunicações Móveis oferecido pelo Departamento de Engenharia de Comunicações da UFRN.}
}

% The paper headers
\markboth{DCO1020 - Comunicações Móveis}{DCO1020 - Comunicações Móveis}

% make the title areahttps://www.overleaf.com/project/5cd4d9c52ee633602b29ee85
\maketitle

\begin{abstract}

Este trabalho apresenta uma calculadora para determinar a taxa de transmissão de downlink do LTE release 10 em diante. A calculadora suporta os principais parâmetros da camada PHY desse protocolo.

\end{abstract}

\begin{keywords}
LTE, LTE Release 10, MIMO, camada PHY.
\end{keywords}

%\IEEEpeerreviewmaketitle

\section{Introdução}

A evolução das comunicações móveis proporcionado diferentes gerações de tecnologia com diversas melhorias ao usuário. A release 10 do LTE, adotada como tecnologia de quarta geração (4G) trouxe enormes melhoras em qualidade de serviço e taxa de transmissão. Essa geração ultrapassou suas antecessores.

\subsection{Proposta do Trabalho}

A calculadora aqui proposta apresenta parâmetros da camada PHY, os quais podem ser variados e, automaticamente, atualizar o valor final de taxa de transmissão do downlink.

Os parâmetros escolhidos foram:
\begin{itemize}
    \item Banda disponível em um componente de portadora (CC): 1.4MHz, 3MHz, 5MHz, 10MHz, 15MHz e 20MHz;
    \item Valor do MCS: de 0 a 28, dado que MCSs 29 a 31 são reservados;
    \item Quantidade de componentes de banda agregados: 1 (sem agregação), 2, 3, 4 e 5;
    \item Tamanho do prefixo cíclico: Normal ou extendido;
    \item Esquema de MIMO: Sem MIMO, MIMO 2x2, MIMO 4x4 e MIMO 8x8.
\end{itemize}

\subsection{Metodologia}

Analisando-se a norma \cite{lte} para a release 10, foi possível determinar o impacto de cada parâmetro citado.

A tabela \ref{tab:tbs-idx} mostra a relação do índice MCS com a ordem de modulação e o respectivo índice TBS, ela também está presente em \cite{lte}, sendo a tabela 7.1.7.1-1.

Com tais informações disponíveis, faz-se necessário determinar o respectivo bloco de recurso físico (PRB, sigla em inglês). Esses valores podem ser observados em \ref{tab:prb}.

A tabela 7.1.7.2.1-1, presente na mesma norma, apresenta o tamanho do respectivo TBS, mostrando os possíveis 2.970 valores associados ao $I_{TBS}$ e ao $N_{PRB}$. Por questão do espaço requerido para tal tabela, ela está sendo omitida desse relatório.

\begin{table}[h!]
    \centering
    \begin{tabular}{r|r|r}
        \makecell[c]{\textbf{Índice MCS}\\ $I_{MCS}$} & \makecell[c]{Ordem de modulação\\$Q_{m}$} & \makecell[c]{Índice TBS\\$I_{TBS}$} \\
        \hline
        0 & 2 & 0\\
        1 & 2 & 1\\
        2 & 2 & 2\\
        3 & 2 & 3\\
        4 & 2 & 4\\
        5 & 2 & 5\\
        6 & 2 & 6\\        
        7 & 2 & 7\\ 
        8 & 2 & 8\\
        9 & 2 & 9\\
        10 & 4 & 9\\
        11 & 4 & 10\\
        12 & 4 & 11\\
        13 & 4 & 12\\
        14 & 4 & 13\\
        15 & 4 & 14\\        
        16 & 4 & 15\\ 
        17 & 6 & 15\\ 
        18 & 6 & 16\\
        19 & 6 & 17\\
        20 & 6 & 18\\        
        21 & 6 & 19\\ 
        22 & 6 & 20\\ 
        23 & 6 & 21\\
        24 & 6 & 22\\        
        25 & 6 & 23\\ 
        26 & 6 & 24\\ 
        27 & 6 & 25\\ 
        28 & 6 & 26
        \end{tabular}
    \caption{Relação entre o índice MCS, a ordem de modulação e o índice TBS}
    \label{tab:tbs-idx}
\end{table}

\begin{table}[h!]
    \centering
    \begin{tabular}{|l|r|r|r|r|r|r|}
        \hline
        Largura de banda & 1.4MHz & 3MHz & 5MHz & 10MHz & 15MHz & 20MHz \\
        \hline
        \makecell[l]{Duração do\\subframe} & \multicolumn{6}{c|}{1ms} \\
        \hline
        \makecell[l]{Espaço entre \\subportadoras} & \multicolumn{6}{c|}{15KHz} \\
        \hline
        Tamanho da FFT & 128 & 256 & 512 & 1024 & 1536 & 2048\\
        \hline
        \makecell[l]{Número de\\subportadoras\\ocupadas} & 72 & 150 & 300 & 600 & 900 & 1200\\
        \hline
        \makecell[l]{Número de\\blocos de\\recurso} & 6 & 18 & 25 & 50 & 75 & 100\\
        \hline
        \makecell[l]{Número de\\símbolos OFDM\\por slot} & \multicolumn{6}{c|}{7 ou 6} \\
        \hline
        \end{tabular}
    \caption{Características da camada física do LTE release 10 para diferentes bandas}
    \label{tab:prb}
\end{table}
\subsubsection{Análise numérica}
Com os parâmetros até então apresentados, podemos aplicar seus valores às equações seguintes. \\

A equação \ref{eq:theoretic-dl} apresenta a taxa de transmissão teórica do downlink enquanto a equação \ref{eq:dl} apresenta a fórmula utilizada para calcular a taxa a partir das tabelas disponibilizadas. \\

Nessas equações, Q representa o índide de modulação apresentado na tabela \ref{tab:tbs-idx}, enquanto $prefix$ representa o tamanho do prefixo cícilico, PRB é determinado pela banda e CC representa a quantidade de componentes de agregação utilizados. Por último, $tbs\_factor$ é um fator que depende do tamanho do prefixo cícilico, podendo ser $\frac{7}{7}$ ou $\frac{6}{7}$
\begin{equation}
   DL_{teorico} = ((12\cdot Q \cdot prefix \cdot mimo \cdot 2 \cdot PRB \cdot 0.75)\cdot\frac{CC}{1000})
   \label{eq:theoretic-dl}
\end{equation}
\begin{equation}
   DL = (mimo \cdot tbs\_factor\cdot TBS\cdot \frac{CC}{1000})
   \label{eq:dl}
   \end{equation}
\section{Desenvolvimento}
\begin{figure}[h!]
    \centering
    \includegraphics[scale=0.2]{calculadora}
    \caption{Exemplo da calculadora para valores de entrada arbitrários}
    \label{fig:trial1-estimated}
\end{figure}
\newpage
\section{Conclusões}
\vspace*{5cm}

Esse trabalho foi bem desafiador, uma vez que colocou a turma em contato direto com uma norma técnica, que se mostrou extensa e com muita informação.

Dessa forma, foi-se necessário ter a habilidade de procurar as informações necessárias para esse trabalho.

Infelizmente não foi possível dedicar tempo suficiente para que o relatório pudesse estar no nível merecido para esse trabalho.

Contudo, montar a calculadora foi uma atividade divertida e engrandecedora. Através da sua criação, pude aprender mais sobre alguns aspectos do desenvolvimento web que não são minha especialidade, como manipulacão do estilo de páginas através de CSS.

Também foi desafiador montar o arco que se atualizasse de acordo com a taxa de downlink encontrada e a máxima taxa possível com a variação dos parâmetros.
\bibliographystyle{IEEEtran}
\bibliography{some_books,some_papers,some_urls}

\end{document}
