\documentclass[journal,11pt,twocolumn]{IEEEtran}
\usepackage[portuguese]{babel}
\usepackage[utf8]{inputenc}
\usepackage[dvips]{graphicx}
\usepackage{amsmath,amsfonts,amssymb}
\usepackage{makecell}
\usepackage[normal]{subfigure}
\usepackage{array,colortbl}
\usepackage{amsmath}
\usepackage{mathrsfs}
\usepackage[sort]{cite}
\usepackage[none]{hyphenat}
\usepackage{minted}

% Those packages were originally here, but I don't think we are gonna need them, so I commented them out
%\usepackage[center,small]{caption}
%\usepackage{color}
%\usepackage{colortbl}
%\usepackage{comment}
%\usepackage{float}
%\usepackage{flushend}
%\usepackage[mathfrak]{mathpi}
%\usepackage{morefloats}
%\usepackage{multicol}
%\usepackage{multirow}
%\usepackage{psfrag}
%\usepackage{rangecite}
%\usepackage{tabularx}
%\usepackage{threeparttable}

\setminted{breaklines=true}

\sloppy
\newcommand{\goodgap}{%
\hspace{\subfigtopskip}%
\hspace{\subfigbottomskip}}
\newcounter{mytempeqncnt}

\begin{document}

% paper title
\title{Projeto I: Caracteriza\c{c}\~{a}o de canais banda estreita}


\author{Meu Nome Vai Aqui
    \thanks{Este trabalho foi como parte do curso de Comunicações Móveis oferecido pelo Departamento de Engenharia de Comunicações da UFRN.}
}

% The paper headers
\markboth{Nome do Evento}{Nome do Evento}

% make the title area
\maketitle

\begin{abstract}
Esse trabalho realiza a decomposição dos efeitos de perda de potência em um sinal em relação à distância entre o transmissor e o receptor. Os dados analisados foram gerados sinteticamente e, portanto, a abordagem descrita é de engenharia reversa. Tenta-se, então, encontrar os parâmetros de descrição do canal ao separar os diferentes efeitos de perda sobre o sinal.
\end{abstract}

\begin{keywords}
Caracterização de canal, aderência estatística, desvanecimento, sombreamento.
\end{keywords}

%\IEEEpeerreviewmaketitle

\section{Introdução}
Caracterização de canais de comunicação é uma tarefa extremamente importante para tornar possível simulações e prototipagem. Embora haja momentos em que simulações e protótipos se encontrem, essa possível reproducibilidade faz com que se possa analisar e depurar problemas, reduzindo os custos relacionados a suas correções e respectivos testes. 

Contudo, uma onda eletromagnética propagando-se está sob efeito de diversos fenômenos físicos. Sendo assim, além de analisar as perdas de potência como um todo, deve-se analisar e conhecer-se os diferentes efeitos que estão influenciando nesse comportamento.

\subsection{Proposta do Trabalho}
Observando-se um sinal inventado, aplicar-se-á uma regressão linear em conjunto com um filtro passa-baixa para separar os diferentes componentes do sinal. Isolando-os devidamente e para melhor caracterização, utilizar-se-á um teste de aderência estatística para modelar as perdas que aparentam comportamento não-determinístico.

\subsection{Metodologia}

Nos últimos anos,...

\subsubsection{Análise numérica}
Todas as análise forma feitas utilizando o seguinte código:
\inputminted[linenos=true]{python}{code/example.py}

Nos últimos anos,...

\subsubsection{Simulação}

Nos últimos anos,...

\subsubsection{Rede Real}

Nos últimos anos,...

\section{Conclusões}
Nos últimos anos...

Nos muitos últimos anos...

Vou enrolar até conseguir duas colunas...

Para isso, preciso ficar adicionando essas linhas todas...

Até conseguir meu objetivo para que esse projeto sirva de exemplo...

E que seja possível apenas copiá-lo para fazermos nossas cópias...

Finalmente, algum conteúdo foi parar na segunda coluna...

\section*{Agradecimentos}
We would like to thank Prof. Luiz Gonzaga for participating in several useful discussions that helped to improve this work.

\bibliographystyle{IEEEtran}
\bibliography{some_books,some_papers}

\end{document}
