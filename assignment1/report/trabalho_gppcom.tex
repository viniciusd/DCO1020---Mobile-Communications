% Duas colunas
%\documentclass[journal,a4paper,11pt]{IEEEtran}

% Uma coluna
\documentclass[journal,11pt,twocolumn]{IEEEtran}
\usepackage[dvips]{graphicx}
\usepackage{amsmath,amsfonts,amssymb}
%\usepackage[mathfrak]{mathpi}
\usepackage{threeparttable}
\usepackage{multicol}
\usepackage{multirow}
\usepackage[normal]{subfigure}
\usepackage{array,colortbl,multicol}
\usepackage{morefloats}
\usepackage{float}
\usepackage{mathrsfs}
\usepackage{comment}
\usepackage{color}
%\usepackage{flushend}
%\usepackage{rangecite}
\usepackage[sort]{cite}
\usepackage[none]{hyphenat}
\usepackage{psfrag}
\usepackage{tabularx}
\usepackage{colortbl}
\usepackage{datetime}
\usepackage[latin1]{inputenc}
\usepackage[center,small]{caption}


\sloppy
%\renewcommand{\baselinestretch}{1.5}
%\renewcommand{\hdashlinewidth}{2pt}
\newcommand{\goodgap}{%
\hspace{\subfigtopskip}%
\hspace{\subfigbottomskip}}
%\newtheorem{struct_type}{struct_title}[in_counter]
\newcounter{mytempeqncnt}

%\usepackage[left=1.50cm, right=1.50cm, top=2.50cm, bottom=1.50cm]{geometry}

\begin{document}

% paper title
\title{T�tulo do Trabalho}


\author{Vicente A. de Sousa~Jr.,
        Fuad Mousse Abinader Jr.,\thanks{Este trabalho foi realizado no Grupo de Pesquisa em Prototipagem R�pida de Solu��es para Comunica��o (GppCom-UFRN).}}

% The paper headers
\markboth{Nome do Evento}{Nome do Evento}

% make the title area
\maketitle

\begin{abstract}

Nos �ltimos anos, temos que usar templates para unificar a linguagem t�cnica~\cite{BOOK_LAW2000}.

\end{abstract}

\begin{keywords}
R�dio Cognitivo, Compartilhamento de Espectro, RRM.
\end{keywords}

%\IEEEpeerreviewmaketitle

\section{Introdu��o}

Nos �ltimos anos,...

\subsection{Proposta do Trabalho}

Nos �ltimos anos,...

\subsection{Metodologia}

Nos �ltimos anos,...

\subsubsection{An�lise num�rica}

Nos �ltimos anos,...

\subsubsection{Simula��o}

Nos �ltimos anos,...

\subsubsection{Rede Real}

Nos �ltimos anos,...

\section{Conclus�es}
\label{sec_concl}

Nos �ltimos anos,...

\section*{Agradecimentos}
We would like to thank Prof. Luiz Gonzaga for participating in several useful discussions that helped to improve this work.

\bibliographystyle{IEEEtran}
\bibliography{gppcom_books,gppcom_standards,gppcom_papers}

\end{document}
